\documentclass{article}

\usepackage{amssymb, amsmath, amsfonts, amsthm}
\usepackage[margin=1.0in]{geometry}
\usepackage{tikz-cd}

\newcommand{\Frac}{\mathrm{Frac}}
\newcommand{\Hom}[2]{\mathrm{Hom}_R(#1, #2)}
\newcommand{\Ext}{\mathrm{Ext}_R}
\DeclareMathOperator{\im}{im}
\DeclareMathOperator{\coker}{coker}
\newcommand{\id}{\mathrm{id}}
\newcommand{\Ann}{\mathrm{Ann}}
\newcommand{\N}{\mathbb{N}}
\newcommand{\Z}{\mathbb{Z}}

\newtheorem{claim}{Claim}

\title{Exercise sheet 6 - Bonus exercise}
\author{Matthew Dupraz}

\begin{document}
	
\maketitle

\subsection*{(1)}

We define $I^k$ and $i^k: I^k \to I^{k+1}$ inductively.

By the Lemma, there exists an injective $R$-module homomorphism
$i^{-1}: N \to I^0$, where $I^0$ is an injective $R$-module.
Now, suppose we have defined $I^k$ and $i^{k-1}$
for some $k \in \N$, then by the Lemma, there exists an injective
$R$-module homomorphism $\hat{i}^k: I^k/\im i^{k-1} \to I^{k+1}$,
where $I^{k+1}$ is an injective $R$-module.
By composing $\hat{i}^k$ with the quotient map
$q_k: I^k \to I^k/\im i^{k-1}$, we obtain an
$R$-homomorphism $i^k: I^k \to I^{k+1}$.
We have that $\ker{i^k} = \ker(\hat{i}^k \circ q_i)
= q_i^{-1}(\ker \hat{i}^k) = q_i^{-1}(0) = \im i^{k-1}$.
Hence we have that the sequence
\begin{equation*}
	% https://tikzcd.yichuanshen.de/#N4Igdg9gJgpgziAXAbVABwnAlgFyxMJZABgBpiBdUkANwEMAbAVxiRGJAF9T1Nd9CKAIzkqtRizYA5LjxAZseAkQBMo6vWatEIAJIA9Dt16KBRAMzrxWtgaGyT-ZSgAsVzZJ0AdLwGMoEDgInGIwUADm8ESgAGYAThAAtkhkIDgQSELGIPFJmdTpSGrWniBY+sAAtFkOOQnJiMWFiJYl2mWGtbkNrc1ubWzl9iGcQA
\begin{tikzcd}
0 \arrow[r] & N \arrow[r, "i^{-1}"] & I^0 \arrow[r, "i^0"] & I^1 \arrow[r, "i^1"] & \cdots
\end{tikzcd}
\end{equation*}
is exact and so $I^\bullet$ is an injective resolution of $N$.

\subsection*{(2)}

Suppose $I$ is an injective $R$-module. Let
$0 \to A \xrightarrow{\alpha} B \xrightarrow{\beta} C \to 0$ be 
a short exact sequence of $R$-modules. Applying
$\Hom{-}{I}$, we get
\begin{equation*}
% https://tikzcd.yichuanshen.de/#N4Igdg9gJgpgziAXAbVABwnAlgFyxMJZABgBpiBdUkANwEMAbAVxiRGJAF9T1Nd9CKAIzkqtRizYAdKQAkIAW2ABBTsACSnLjxAZseAkQDMo6vWatEIGfKUBhNZu299AogBZT4i2w7cX-IYoAExe5pJWNorAAEKOWpxiMFAA5vBEoABmAE6KSCYgOBBIwf4gOXmIoYXFiJ7eEdZy0TIARjA4dPEg1Ax07QwACnwGgiDZWCkAFjjO5bkKSPVFSCINlk22wDKMaFNdGmpavf0wQyNuVhPTs2UVi4hrK4jEiZxAA
\begin{tikzcd}[row sep=large,column sep=huge]
0 & \Hom{A}{I} \arrow[l] & \Hom{B}{I} \arrow[l, "\Hom{\alpha}{I}{}"'] & \Hom{C}{I} \arrow[l, "\Hom{\beta}{I}"'] & 0 \arrow[l]
\end{tikzcd}
\end{equation*}
To show this sequence is exact, it suffices to show
$\Hom{\alpha}{I}$ is surjective, as we know $\Hom{-}{I}$ is
left-exact.
Let $\gamma \in \Hom{A}{I}$,
then by injectivity of $I$, there exists
an $\eta: B \to I$ making the following diagram commute
\begin{equation*}
% https://tikzcd.yichuanshen.de/#N4Igdg9gJgpgziAXAbVABwnAlgFyxMJZABgBpiBdUkANwEMAbAVxiRGJAF9T1Nd9CKAIzkqtRizYBBLjxAZseAkQBMo6vWatEIAEKzeigURFCxmyToCSXMTCgBzeEVAAzAE4QAtkjIgcEEhC3G6ePogi-oGIav50WAxsABYQEADWIBoS2iAAOrmMaEl0BiAe3kHUAUgAzFlabPkOdF5eJSFlYUix1Yh1IAxYYDlQdHBJ9pniDTr5MDjtFJxAA
\begin{tikzcd}
0 \arrow[r] & A \arrow[r, "\alpha", hook] \arrow[d, "\gamma"] & B \arrow[ld, "\eta", dashed] \\
            & I                                               &                             
\end{tikzcd}	
\end{equation*}
In other words $\gamma = \eta \circ \alpha = 
\Hom{\alpha}{I}(\eta)$ and so $\Hom{\alpha}{I}$ is surjective,
which shows that $\Hom{-}{I}$ is an exact functor.

Now, suppose $\Hom{-}{I}$ is an exact functor. We will show $I$
is an injective $R$-module. Let $\alpha: A\to B$ be an injective
$R$-morphism and $\gamma: A \to I$ an $R$-morphism.
We have that
$0 \to A \xrightarrow{\alpha} B \xrightarrow{q} B/\alpha(A) \to 0$
is a short exact sequence. Since $\Hom{-}{I}$ is an exact functor,
we obtain that
\begin{equation*}
% https://tikzcd.yichuanshen.de/#N4Igdg9gJgpgziAXAbVABwnAlgFyxMJZABgBpiBdUkANwEMAbAVxiRGJAF9T1Nd9CKAIzkqtRizYAdKQAkIAW2ABBTsACSnLjxAZseAkQDMo6vWatEIGfKUBhNZu299AogBZT4i2w7cX-IYoAExe5pJWNorAAEKOWpxiMFAA5vBEoABmAE6KSCYgOBBIwf4gOXmIoYXFiJ7eEdZy0TIARjA4dPEg1Ax07QwACnwGgiDZWCkAFjjO5bkKSPVFSCINlk22wDKMaFNdGmpavf0wQyNuVhPTs2UVi4hrK4jEiZxAA
\begin{tikzcd}[row sep=large,column sep=huge]
0 & \Hom{A}{I} \arrow[l] & \Hom{B}{I} \arrow[l, "\Hom{\alpha}{I}{}"'] &
\Hom{B/\alpha(A)}{I} \arrow[l, "\Hom{q}{I}"'] & 0 \arrow[l]
\end{tikzcd}
\end{equation*}
is exact and hence $\Hom{\alpha}{I}$ is surjective. So there
exists some $\eta \in \Hom{B}{I}$, such that
$\gamma = \Hom{\alpha}{I}(\eta) = \eta \circ \alpha$.
So we get that the following diagram commutes:
\begin{equation*}
% https://tikzcd.yichuanshen.de/#N4Igdg9gJgpgziAXAbVABwnAlgFyxMJZABgBpiBdUkANwEMAbAVxiRGJAF9T1Nd9CKAIzkqtRizYBBLjxAZseAkQBMo6vWatEIAEKzeigURFCxmyToCSXMTCgBzeEVAAzAE4QAtkjIgcEEhC3G6ePogi-oGIav50WAxsABYQEADWIBoS2iAAOrmMaEl0BiAe3kHUAUgAzFlabPkOdF5eJSFlYUix1Yh1IAxYYDlQdHBJ9pniDTr5MDjtFJxAA
\begin{tikzcd}
0 \arrow[r] & A \arrow[r, "\alpha", hook] \arrow[d, "\gamma"] & B \arrow[ld, "\eta", dashed] \\
            & I                                               &                             
\end{tikzcd}	
\end{equation*}
and hence $I$ is injective.

\subsection*{(3)}

To see this diagram commute, it suffices to show every square
commutes. The lower left square commutes trivially, the other
squares are of the form
\begin{equation*}
	% https://tikzcd.yichuanshen.de/#N4Igdg9gJgpgziAXAbVABwnAlgFyxMJZABgBpiBdUkANwEMAbAVxiRAB12AJCAW2AAKAfWABrALQBGAL7TgASSGjpIaaXSZc+QijKSqtRizace-YWKmyFIiTJVqN2PASKTyB+s1aIO3PoJKcorKquogGM7abqT61F7GvqYBwso2lvaqBjBQAObwRKAAZgBOfEjuIDgQSGSG3ibssAw4dLZSpAAEdg7hpeWIddVIAEzxRj4gUO3u3b3FZbxIAMzUw4hj9Yl+za22XT1hCwOV66tbk9MZB1ZZ0kA
\begin{tikzcd}
\Hom{P_{k-1}}{I^k} \arrow[r, "{d_{k-1, k}}"]
& \Hom{P_k}{I^k}                                    \\
\Hom{P_{k-1}}{I^{k-1}} \arrow[u, "{\delta_{k-1, k-1}}"] \arrow[r, "{d_{k-1,
k-1}}"] & \Hom{P_k}{I^{k-1}} \arrow[u, "{\delta_{k, k-1}}"]
\end{tikzcd}
\end{equation*}
Where we define $P_{-1} = M$ and $I^{-1} = N$.
This diagram commutes, since for any
$\phi \in \Hom{P_{k-1}}{I^{k-1}}$, 
$$d_{k-1, k} \circ \delta_{k-1, k-1}(\phi) =
(\iota^{k-1} \circ \phi) \circ p_k
= \iota^{k-1} \circ (\phi \circ p_k)
= \delta_{k, k-1} \circ \delta_{k-1, k-1}(\phi)
$$
Hence the diagram commutes.

Before showing that the rows and columns which are not blue
are exact, let's
first show two claims, for which I am not sure whether we proved
them in class or not
\begin{claim}
	If $\{0 \to B_k \xrightarrow{\iota_k} A_k
	\xrightarrow{\beta_{k-1}} B_{k-1} \to 0\}_{k \in \Z}$
	is a collection of short exact sequences, then
	$$\cdots \xrightarrow{\alpha_k} A_k \xrightarrow{\alpha_{k-1}}
	A_{k-1} \xrightarrow{\alpha_{k-2}}
	A_{k-2} \xrightarrow{\alpha_{k-3}} \cdots $$
	where $\alpha_k = \iota_k \circ \beta_k$, is an exact sequence.
\end{claim}
\begin{proof}
	We have that for all $k \in \Z$
	$$\im \alpha_k = \im(\iota_k \circ \beta_k) = \im \iota_k
	= \ker \beta_{k-1} = \ker(\iota_{k-1} \circ \beta_{k-1})
	= \ker \alpha_{k-1}$$.
	We used the fact that $\iota_k$ is injective 
	and $\beta_k$ is surjective for all $k \in \Z$.
\end{proof}

\begin{claim}
	If a functor $\mathrm{Mod}_R \to \mathrm{Mod}_R$ is exact,
	it preserves exactness of long exact sequences.
\end{claim}
\begin{proof}
If the sequence
$$\cdots \xrightarrow{\alpha_k} A_k \xrightarrow{\alpha_{k-1}}
A_{k-1} \xrightarrow{\alpha_{k-2}}
A_{k-2} \xrightarrow{\alpha_{k-3}} \cdots $$
is exact, then $0 \to \ker(\alpha_k) \xrightarrow{\iota_k} A_k
\xrightarrow{\beta_k} \ker(\alpha_{k-1}) \to 0$ is exact
for all $k \in \Z$, where
$\iota_k$ is the inclusion map and
$\beta_k$ is just $\alpha_{k-1}$ corestricted to
$\ker(\alpha_{k-1})$. If $F: \mathrm{Mod}_R \to \mathrm{Mod}_R$
is exact, then 
$$0 \to F(\ker(\alpha_k)) \xrightarrow{F\iota_k} F(A_k)
\xrightarrow{F\beta_k} F(\ker(\alpha_{k-1})) \to 0$$
is exact for all $k \in \Z$ and so by Claim 1,
$$\cdots \xrightarrow{F\iota_k\circ F\beta_k}
F(A_k) \xrightarrow{F\iota_{k-1}
\circ F\beta_{k-1}}
F(A_{k-1}) \xrightarrow{F\iota_{k-2}\circ F\beta_{k-2}} \cdots$$
is exact. Furthermore, since $F$ is a functor,
$F\iota_k\circ F\beta_k = F(\iota_k \circ \beta_k) = F\alpha_k$ for
all $k$ and hence we deduce
$$\cdots \xrightarrow{F\alpha_k}
F(A_k) \xrightarrow{F\alpha_{k-1}}
F(A_{k-1}) \xrightarrow{F\alpha_{k-2}} \cdots$$
is exact. We get the same result for contravariant functors.
\end{proof}

Now, since $\Hom{-}{I}$ is exact when $I$ is injective, and
$\Hom{P}{-}$ is exact when $P$ is projective (Remark 5.4.2), we get
by Claim 2, that the rows and columns of the diagram which are
not blue are exact, since they are images of exact sequences by
these two functors.

\subsection*{(4)}

We have that $H^0(\Hom{M}{I^\bullet}) = \ker \delta_{-1, 0}$
and $H^0(\Hom{P_\bullet}{N}) = \ker d_{0, -1}$.
\begin{equation*}
	\ker \delta_{-1, 0} = \ker(d_{-1, 1}\circ \delta_{-1, 0})
	= \ker(\delta_{0, 0} \circ d_{-1, 0})
	= d_{-1, 0}^{-1}(\ker \delta_{0, 0})
\end{equation*}
We used that $d_{-1, 1}$ is injective, since rows are exact and
the commutativity of the diagram. We also have that
\begin{equation*}
	\ker d_{0, -1} = \ker(\delta_{1, -1}\circ d_{0, -1})
	= \ker(d_{0, 0} \circ \delta_{0, -1})
	= \delta_{0, -1}^{-1}(\ker d_{0, 0})
\end{equation*}
Similarly, we used the injectivity of $\delta_{1, -1}$.
Hence we have that
\begin{align*}
	d_{-1, 0}(H^0(\Hom{M}{I^\bullet}))
	&= d_{-1, 0}(d_{-1, 0}^{-1}(\ker \delta_{0, 0}))
	= \ker \delta_{0, 0} \cap \im d_{-1, 0}\\
	&= \ker{\delta_{0, 0}} \cap \ker d_{0, 0}
	= \im \delta_{0, 1} \cap \ker d_{0, 0}\\
	&= \delta_{0, -1}(\delta_{0, -1}^{-1}(\ker(d_{0, 0})))
	= \delta_{0, -1}(H^0(\Hom{P_\bullet}{N}))
\end{align*}
Since $d_{-1, 0}$ and $\delta_{0, -1}$ are injective, we get that
\begin{equation*}
	H^0(\Hom{M}{I^\bullet}) \cong
	d_{-1, 0}(H^0(\Hom{M}{I^\bullet}))
	= \delta_{0, -1}(H^0(\Hom{P_\bullet}{N}))
	\cong H^0(\Hom{P_\bullet}{N})
\end{equation*}

\subsection*{(5)}

Let $C^0 := \Hom{P_0}{I^0}$ and
$C^1 := \Hom{P_1}{I^0} \oplus \Hom{P_0}{I^1}$. We also define
$\Delta^0: C^0 \to C^1$ by 
$\Delta^0(x) = (d_{0, 0}(x), \delta_{0, 0}(x))$.
Now, let
\begin{align*}
	\alpha: H^1(\Hom{M}{I^\bullet}) =
	\ker \delta_{-1, 1}/\im \delta_{-1, 0} &\to \coker(\Delta^0)
	= C^1/\im \Delta^0\\
	[x] &\mapsto [(0, d_{-1, 1}(x))]
\end{align*}
then $\alpha$ is well defined, indeed, suppose 
$[x] = [y] \in H^1(\Hom{M}{I^\bullet})$, then
let $z \in \Hom{M}{I^0}$ such that $x - y = \delta_{-1, 0}(z)$.
So we have that
\begin{align*}
	(0, d_{-1, 1}(x)) - (0, d_{-1, 1}(y)) &= 
	(0, d_{-1, 1}(x - y))\\
	&= (0, d_{-1, 1}\circ\delta_{-1, 0}(z))\\
	&= (0, \delta_{0, 0}\circ d_{-1, 0}(z))\\
	&= (d_{0, 0}\circ d_{-1, 0}(z),
	\delta_{0, 0}\circ d_{-1, 0}(z))\\
	&= \Delta^0(d_{-1, 0}(z)) \in \im \Delta^0
\end{align*}
and hence $[(0, d_{-1, 1}(x))] = [(0, d_{-1, 1}(y))]$ and hence
$\alpha$ is well defined.

We will now show that $\alpha$ is injective. Let $[x], [y]
\in H^1(\Hom{M}{I^\bullet})$ such that
$\alpha([x]) = \alpha([y])$. Then
\begin{align*}
	[(0, d_{-1, 1}(x))] = [(0, d_{-1, 1}(y))]
	&\implies (0, d_{-1, 1}(x - y)) \in \im \Delta^0\\
	&\implies d_{-1, 1}(x - y) \in \im \delta_{0, 0}
	= \ker \delta_{0, 1}\\
	&\implies x - y \in \ker \delta_{0, 1} \circ d_{-1, 1}
	= \ker d_{-1, 0} \circ \delta_{-1, 1} 
	= \ker \delta_{-1, 1} = \im \delta_{-1, 0}
\end{align*}
and so $[x] = [y]$, which shows $\alpha$ is injective.

Similarly, we define
\begin{align*}
	\beta: H^1(\Hom{P_\bullet}{N}) =
	\ker d_{1, -1}/\im d_{0,-1} &\to \coker(\Delta^0)
	= C^1/\im \Delta^0\\
	[x] &\mapsto [(\delta_{1, -1}(x), 0)]
\end{align*}
and as before, we show $\beta$ is well defined. Indeed, suppose 
$[x] = [y] \in H^1(\Hom{P_\bullet}{N})$, then
let $z \in \Hom{P_0}{N}$ such that $x - y = d_{0, -1}(z)$.
So we have that
\begin{align*}
	(\delta_{1, -1}(x), 0) - (\delta_{1, -1}(y), 0) &= 
	(\delta_{1, -1}(x - y), 0)\\
	&= (\delta_{1, -1}\circ d_{0, -1}(z), 0)\\
	&= (d_{0, 0}\circ \delta_{0, -1}(z), 0)\\
	&= (d_{0, 0}\circ \delta_{0, -1}(z),
	\delta_{0, 0}\circ \delta_{0, -1}(z))\\
	&= \Delta^0(\delta_{0, -1}(z)) \in \im \Delta^0
\end{align*}
and hence $[(\delta_{1, -1}(x), 0)] = [(\delta_{1, -1}(y), 0)]$
and hence
$\beta$ is well defined.

We will now show that $\beta$ is injective. Let $[x], [y]
\in H^1(\Hom{P_\bullet}{N})$ such that
$\beta([x]) = \beta([y])$. Then
\begin{align*}
	[(\delta_{1, -1}(x), 0)] = [(\delta_{1, -1}(y), 0)]
	&\implies (\delta_{1, -1}(x - y), 0) \in \im \Delta^0\\
	&\implies \delta_{1, -1}(x - y) \in \im d_{0, 0}
	= \ker d_{1, 0}\\
	&\implies x - y \in \ker d_{1, 0} \circ \delta_{1, -1}
	= \ker \delta_{0, -1} \circ d_{1, -1} 
	= \ker d_{1, -1} = \im d_{0, -1}
\end{align*}
and so $[x] = [y]$, which shows $\beta$ is injective.


Let's now show that $\im \alpha = \im \beta$.
Let $\alpha([x]) \in \im \alpha$, then $\alpha([x])
= [(0, d_{-1, 1}(x))]$
We have that $\delta_{0, 1} \circ d_{-1, 1}(x)
= d_{-1, 2}\circ \delta_{-1, 1}(x) = 0$, since
$x \in \ker \delta_{-1, 1}$, and hence $d_{-1, 1}(x) \in 
\ker \delta_{0, 1} = \im \delta_{0, 0}$, so let $y \in 
\Hom{P_0}{I^0} = C^0$ such that $d_{-1, 1}(x) = \delta_{0, 0}(y)$.
We have that
\begin{equation*}
	[(0, d_{-1, 1}(x))] =
	[(0, d_{-1, 1}(x)) - \Delta^0(y)] =
	[(-d_{0, 0}(y), 0)]
\end{equation*}
Furthermore,
\begin{equation*}
	\delta_{1, 0}\circ d_{0, 0}(y) = d_{0, 1} \circ \delta_{0,0}(y)
	= d_{0, 1} \circ d_{-1, 1}(x) = 0
\end{equation*}
hence $d_{0, 0}(y) \in \ker \delta_{1, 0} = \im \delta_{1, -1}$
and so there exists $z \in \Hom{P_1}{N}$ such that
$d_{0, 0}(y) = \delta_{1, -1}(z)$.
Hence we deduce
\begin{equation*}
	\alpha([x]) = [(0, d_{-1, 1}(x))] = 
	[(-d_{0, 0}(y), 0)] = [(-\delta_{1, -1}(z), 0)]
	= \beta([-z])
\end{equation*}
and so $\im \alpha \subseteq \im \beta$.
Using a symmetric reasoning, we also deduce
$\im \beta \subseteq \im \alpha$ and hence
$\im \alpha = \im \beta$:

Let $\beta([x]) \in \im \beta$, then $\beta([x])
= [(\delta_{1, -1}(x), 0)]$
We have that $d_{1, 0} \circ \delta_{1, -1}(x)
= \delta_{2, -1}\circ d_{1, -1}(x) = 0$, since
$x \in \ker d_{1, -1}$, and hence $\delta_{1, -1}(x) \in 
\ker d_{1, 0} = \im d_{0, 0}$, so let $y \in 
\Hom{P_0}{I^0} = C^0$ such that
$\delta_{1, -1}(x) = d_{0, 0}(y)$.
We have that
\begin{equation*}
	[(\delta_{1, -1}(x), 0)] =
	[(\delta_{1, -1}(x), 0) - \Delta^0(y)] =
	[(0, -\delta_{0, 0}(y))]
\end{equation*}
Furthermore,
\begin{equation*}
	d_{0, 1}\circ \delta_{0, 0}(y) = \delta_{1, 0} \circ d_{0,0}(y)
	= \delta_{1, 0} \circ \delta_{1, -1}(x) = 0
\end{equation*}
hence $\delta_{0, 0}(y) \in \ker d_{0, 1} = \im d_{-1, 1}$
and so there exists $z \in \Hom{M}{I^1}$ such that
$\delta_{0, 0}(y) = d_{-1, 1}(z)$.
Hence we deduce
\begin{equation*}
	\beta([x]) = [(\delta_{1, -1}(x), 0)] = 
	[(0, -d_{0, 0}(y))] = [(0, -d_{-1, 1}(z))]
	= \alpha([-z])
\end{equation*}
and so $\im \beta \subseteq \im \alpha$ and hence
$\im \alpha = \im \beta$.


Since both $\alpha$ and $\beta$ are injective, we deduce
\begin{equation*}
	H^1(\Hom{M}{I^\bullet}) \cong \im \alpha = \im \beta
	\cong H^1(\Hom{P_{\bullet}}{N})
\end{equation*}

\end{document}
