\documentclass{article}

\usepackage{amssymb, amsmath, amsfonts, amsthm}
\usepackage[margin=1.0in]{geometry}
\usepackage{tikz-cd}

\DeclareMathOperator{\Frac}{Frac}
\DeclareMathOperator{\im}{im}
\DeclareMathOperator{\Hom}{Hom}
\DeclareMathOperator{\Ext}{Ext}
\DeclareMathOperator{\coker}{coker}
\DeclareMathOperator{\id}{id}
\DeclareMathOperator{\Ann}{Ann}

\newcommand{\N}{\mathbb{N}}
\newcommand{\Z}{\mathbb{Z}}
\newcommand{\R}{\mathbb{R}}
\newcommand{\C}{\mathbb{C}}

\title{Exercise sheet 8 - Bonus exercise}
\author{Matthew Dupraz}

\begin{document}
	
\maketitle

\subsection*{(1)}

Let $\frac{p}{q} \in K$, where $p \in R$, $q \in R^\times$.
Let $a := p \prod_{g \in G\setminus\{\id\}} g\cdot q$ and
$b := \prod_{g \in G}g \cdot q$.
We have that for all $g \in G$, $g \cdot q \neq 0$, because $G$ acts
on $K$ by ring automorphisms. Hence $\frac{a}{b} = \frac{p}{q}$.
Furthermore, $a \in R$, since for all $g \in G$, $g \cdot q \in R$,
and $b \in S^\times$, since $b \in R^\times$ and for any $h \in G$,
\begin{equation*}
	h \cdot b = h \cdot \prod_{g \in G} g \cdot q
	= \prod_{g \in G} hg \cdot q
	= \prod_{g' \in hG} g' \cdot q
	= \prod_{g' \in G} g' \cdot q = b,
\end{equation*}
hence $b \in L$. So we have that $\frac{a}{b}$ is of the desired form.

\subsection*{(2)}

Let $\frac{a}{b} \in L$ be in the form specified in (1), i.e. with $a \in R$
and $b \in S^\times$. Since $L = K^G$, we get for all $g \in G$,
\begin{equation*}
	\frac{a}{b} = g \cdot \frac{a}{b} = \frac{g \cdot a}{g \cdot b}
	= \frac{g \cdot a}{b}
\end{equation*}
and hence $g\cdot a = a$. We deduce that $a \in L$ and so $a \in S$.
Hence any element of $L$ can be written in the form $\frac{a}{b}$, where
$a \in S$, $b \in S^\times$. Hence $L \subset \Frac(S)$,
but because $L$ is a field and $\Frac(S)$ is the smallest field containing $S$,
we get that $L = \Frac(S)$.

\subsection*{(3)}

Let $l \in L$, which is integral over $S$, then there exists a monic polynomial
$p(x) \in S[x]$ such that $p(l) = 0$. Hence $l$ is also integral over $R$,
since $p(x)$ can be seen as a monic polynomial in $R[x]$.
But then, since $R$ is integrally closed, $l \in R$ and hence
$l \in R \cap L = S$. We conclude that $S$ is integrally closed.

\subsection*{(4)}

Let $\zeta = e^{2\pi i/n}$. Define
\begin{align*}
	\phi: \C(x, y) &\to \C(x, y)\\
	f &\mapsto f(\zeta x, \zeta y)
\end{align*}
then $\phi$ is an automorphism. Indeed, it is a field homomorphism, since
\begin{equation*}
	\phi(f \cdot g) = (f \cdot g) (\zeta x, \zeta y)
= f(\zeta x, \zeta y) \cdot g (\zeta x, \zeta y)
= \phi(f)\phi(g)
\end{equation*}
and 
\begin{equation*}
	\phi(f + g) = (f + g) (\zeta x, \zeta y)
= f(\zeta x, \zeta y) + g (\zeta x, \zeta y)
= \phi(f) + \phi(g).
\end{equation*}
Furthermore, it is bijective, since $f \mapsto f(\zeta^{-1}x, \zeta^{-1}y)$ is 
its inverse. Hence $\phi$ is an automorphism.

Consider the (natural) action of $G = \langle \phi \rangle$ on $\C(x, y)$.
We have that $|G| = n$, since $o(\phi) = n$ as $\zeta$ is an $n$-th primitive
root of unity.
We will show $\C[x, y]^G = \C[x^n, x^{n-1}y, \dots, y^n] =: S$.
We clearly have that $S \subseteq \C[x, y]^G$, since
$x^n, x^{n-1}y, \dots, y^n$ are all fixed by $\phi$.

Let $p = \sum_{i,j\leq r} a_{ij} x^i y^j \in \C[x, y]^G$
We have that
\begin{equation*}
\sum_{i,j\leq r} a_{ij} x^i y^j = 
\phi\left(\sum_{i,j\leq r} a_{ij} x^i y^j\right)
= \sum_{i,j\leq r} \phi(a_{ij} x^i y^j)
= \sum_{i,j\leq r} \zeta^{i+j}a_{ij} x^i y^j
\end{equation*}
and hence $\forall i,j \in \{0, \dots, r\}$, 
$a_{ij} = \zeta^{i+j}a_{ij}$, from which we deduce $a_{ij} = 0$ if $i+j \not\in 
(n)$. We get that $p \in S$ and so $\C[x, y]^G \subseteq S$.

So we have that $\C[x, y] \cap \C(x, y)^G = \C[x, y]^G = S$. Since $\C[x,y]$ is
a UFD (by the transfer theorem), we get that $\C[x,y ]$ is integrally closed
(Example 6.2.8). Hence using the previous points of this exercise,
we deduce $S$ is also integrally closed.


\end{document}
