\documentclass{article}

\usepackage{amssymb, amsmath, amsfonts, amsthm}
\usepackage[margin=1.0in]{geometry}
\usepackage{tikz-cd}

\DeclareMathOperator{\Frac}{Frac}
\DeclareMathOperator{\im}{im}
\DeclareMathOperator{\Hom}{Hom}
\DeclareMathOperator{\Ext}{Ext}
\DeclareMathOperator{\Tor}{Tor}
\DeclareMathOperator{\coker}{coker}
\DeclareMathOperator{\id}{id}
\DeclareMathOperator{\Ann}{Ann}

\newcommand{\N}{\mathbb{N}}
\newcommand{\Z}{\mathbb{Z}}
\newcommand{\R}{\mathbb{R}}
\newcommand{\C}{\mathbb{C}}

\newcommand{\tensor}{\otimes}
\newcommand{\quotient}[2]{{\raisebox{.2em}{$#1$}\left/\raisebox{-.2em}{$#2$}\right.}}
\newcommand{\isom}{\cong}

\title{Exercise sheet 9 - Bonus exercise}
\author{Matthew Dupraz}

\begin{document}
	
\maketitle

Let $F$ be a field and $I = (f) \subseteq F[x]$ a principal ideal.
We can write $f = x^rg$ for some $r \in \N$ and $h \not\in (x)$.

\subsection*{(1)}

We consider the localization $F[x]_x$. Let $\phi: F[x] \to F[x]_x$ be the
structure homomorphism. We will show that
$I^{ec} = (g)$.

By proposition 7.3.9, we have that
\begin{align*}
	I^{ec} &= \bigcup_{k \in \N} (I : x^k)\\
	&= \bigcup_{k \in \N} \{h \in F[x] : x^kh \in I \}\\
	&= \{h \in F[x] : \exists k \in \N, x^kh \in I\}\\
	&= \{h \in F[x] : \exists k \in \N, x^rg \mid x^kh\}\\
	&= \{h \in F[x] : \exists k \in \N, x^r \mid x^kh
	\textrm{ and } g \mid h\}
\end{align*}
The last equality follows from the fact $g \not\in (x)$ and so $x^r$ 
and $g$ are coprime, hence $x^rg \mid x^kh$ iff $x^r \mid x^kh$ and
$g \mid x^k h$. For the same reason, $g$ and $x^k$ are coprime and so
$g \mid x^k h$ iff $g \mid h$. We have that $x^r \mid x^k h$ for any $k$
large enough and so we get that
\begin{equation*}
	I^{ec} = \{h \in F[x]: g \mid h\} = (g)
\end{equation*}
This implies that $I^{ec} = I$ iff $r = 0$ (that is if $f \not\in (x)$).

\subsection*{(2)}

Now we consider the localization $F[x]_{(x)}$. Let now
$\phi: F[x] \to F[x]_{(x)}$. We will show that $I^{ec} = (x^r)$.

By proposition 7.3.0 we have that
\begin{align*}
	I^{ec} &= \bigcup_{h \in F[x]\setminus(x)} (I : h)\\
	&= \bigcup_{h \in F[x]\setminus(x)} \{p \in F[x] : ph \in I \}\\
	&= \{p \in F[x] : \exists h \in F[x]\setminus(x), ph \in I \}\\
	&= \{p \in F[x] : \exists h \in F[x]\setminus(x), x^rg \mid ph \}\\
	&= \{p \in F[x] : \exists h \in F[x]\setminus(x), g \mid ph,
	x^r \mid p \}
\end{align*}
Again, the last equality follows from the fact $g \not\in (x)$ and so 
$g$ and $x^r$ are coprime. We also have $h \not\in (x)$, hence $h$ and
$x^r$ are coprime, so $x^r \mid ph$ iff $x^r \mid p$.
We have that $g | ph$ for example for $h = g \not\in (x)$, hence we get
that
\begin{equation*}
	I^{ec} = \{p \in F[x] : x^r | p\} = (x^r)
\end{equation*}
This implies that $I^{ec} = i$ iff $g = 0$ (that is if $f = x^r$).
\end{document}
