\documentclass{article}

\usepackage{amsfonts, amssymb, amsmath}

\author{Matthew Dupraz}
\title{Exercise sheet 2 - Bonus}

\newcommand{\N}{\mathbb{N}}
\newcommand{\R}{\mathbb{R}}
\newcommand{\Z}{\mathbb{Z}}

\begin{document}

\maketitle

\subsection*{(1)}

Let $f = f_1\cdots f_n$.
We'll show
$$0 = (f_1\cdots f_n)/(f) \subset
(f_1\cdots f_{n-1})/(f) \subset \dots \subset
(f_1)/(f) \subset R/(f)$$
is the composition series of $R/(f)$, 
which would imply $\mathrm{length}(R/(f)) = n$.

By the correspondence theorem, the submodules of $R/(f)$ are
in bijective correspondence with submodules $M$ of ${_RR}$ such that
$(f) \subseteq M \subseteq {_RR}$, which is induced by the quotient
homomorphism. Since $R$ is a PID, $M \subseteq {_RR}$ is of
the form $(g)$ for $g \in R$. In particular, $g | f$.
Hence, the submodules of $R/(f)$ are exactly the modules
$(g)/(f)$ for $g|f$. Notice that for
$M \subsetneq N \subseteq {_RR}$, we also
have $M/(f) \subsetneq N/(f)$, because the correspondence is
bijective.

This implies that for any $k \in \{1, \dots n\}$, we have that
$(f_1 \cdots f_k)/(f)$ is a maximal submodule of
$(f_1 \cdots f_{k-1})/(f)$.
Indeed, if $(f_1\cdots f_k)/(f) \subsetneq (g)/(f)
\subsetneq (f_1 \cdots f_{k-1})/(f)$ for some $g|f$, then in
particular, $(f_1\cdots f_k) \subsetneq (g)
\subsetneq (f_1 \cdots f_{k-1})$
and so $f_1 \cdots f_{k-1} | g$ and $g | f_1\cdots f_k$.
This implies
that either $g \sim f_1 \cdots f_{k-1}$ or
$g \sim f_1 \cdots f_k$,
both of which are absurd.
This shows that
$$0 = (f_1\dots f_n)/(f) \subsetneq
(f_1\dots f_{n-1})/(f) \subsetneq \dots \subsetneq
(f_1)/(f) \subsetneq R/(f)$$
is a composition series of $R/(f)$.
% continue from here

%then $g | f_1$, but since $f_1$ is prime, this implies either 
%$g \sim f_1$ or $g \sim 1$, either one of which is absurd.
%Now, let $g = f_2 \dots f_{n+1}$,
%we have that $(f_1)/(f) = (f_1)/(f_1)\cap(g) \cong
%(f_1)+(g)/(g)$ by the second
%isomorphism theorem. Since $f_1$ and $g$ are
%relatively prime, we have that
%$(f_1)+(g)/(g) 
%= R/(g)$. By induction hypothesis,
%$R/(g)$ is of length $n$ and since $(f_1)/(f) \cong R/(g)$
%is maximal in $R/(f)$, we deduce $R/(f)$ is of length $n+1$ (since
%we get a decomposition series of $R/(f)$ by just appending $R/(f)$
%to the decomposition series of $R/(g)$, modulo isomorphism).

\subsection*{(2)}

Since $R[x]$ is a PID, if $f \in R[x]$ is irreductible,
then it is prime. Let $f \in R[x]$ be a polynomial with exactly
$n\geq 0$ non-real roots, and let $f=f_1 \cdots f_k$ be a
decomposition of $f$ into irreductible factors.
$f_i$ is either of degree 1 (if it has a real root),
or of degree 2 (if it has two conjugate non-real roots).
Hence we deduce that exactly $n/2$ of the irreductible 
factors of $f$ are of degree 2.
By part (1), $\mathrm{length}_{\R[x]}\left(\R[x]/(f)\right)
= k$. We have seen in Anneaux et Corps, that
$\mathrm{dim}_\R\left(\R[x]/(f)\right) = \mathrm{deg}(f) = 
(n/2)\times2 + (k - n/2)\times1 = k + n/2$.
Hence $\mathrm{dim}_\R\left(\R[x]/(f)\right) -
\mathrm{length}_{\R[x]}\left(\R[x]/(f)\right) = n/2$.

\subsection*{(3)}

Let $M$ be a $\Z$-module. Suppose $M$ is finite. Then any strictly
increasing chain of submodules $M_0 \subsetneq M_1 \subsetneq 
\dots$ is of finite length. Hence $M$ is of finite length.

Now suppose $M$ is infinite. There are two cases:
\begin{enumerate}
	\item $M$ is not finitely generated, then we can construct a
		strict chain of finitely generated submodules of any length,
		which we show by induction on the length $k$.
		For the base case we just let
		$M_0 = 0$. Now suppose we have defined $M_k$ for some $k$.
		Then define $M_{k+1}$ by adjoining an element
		$m_{k+1} \in M\setminus M_k$. By hypothesis $M_k$ is finitely
		generated, so such an element exists. Furthermore,
		$M_{k+1}$ is also finitely generated and
		$M_k \subsetneq M_{k+1}$.

		Since the length of $M$ is strictly larger than the length
		of any strictly increasing chain of submodules, we deduce
		that the length of $M$ has to be infinite.
	\item $M$ is finitely generated, then if $\{m_1, \dots, m_n\}$
		is a generating set, at least for one $i$, the submodule
		$(m_i)$ has to be infinite (since $M$ is infinite;
		alternatively, at least one of the factors that appear in the
		decomposition given by the fundamental theorem of finitely
		generated modules over a PID has to be $\Z$, hence there is
		an element of infinite order).
		Consider the chain of submodules
		$(m_i) \supset (2m_i) \supset (3m_i) \supset \dots$.
		If for any $k$, $(km_i) = ((k+1)m_i)$, then we have that
		$km_i = n(k+1)m_i$ for some $n \in \Z$, in particular, 
		$(k - n(k+1))m_i = 0$. Since $m_i$ is of infinite order, this
		implies $(k - n(k + 1)) = 0$, but that's not satisfied for
		any $k \geq 1$ and $n \in \Z$. Hence for all $k$,
		$(km_i) \supsetneq ((k+1)m_i)$, and so we constructed an
		infinite strictly decreasing chain of submodules.
		This implies $M$ is of infinite length.	
\end{enumerate}

\subsection*{(4)}

We can take for example $\R^2$ as a module over $\R$.
Any 2 linearly independent elements of $\R^2$ generate $\R^2$, hence
any strict chain of submodules of $\R^2$ is of length at most 2.
So $\R^2$ is of finite length. It also has an infinite amount of
submodules, for example $(1, x)\R$ is a submodule for any $x \in \R$
and for $x, y \in \R$ distinct, $(1, x)\R \neq (1, y)\R$.


\end{document}
