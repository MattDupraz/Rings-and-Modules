\documentclass{article}

\usepackage{amssymb, amsmath, amsfonts, amsthm}
\usepackage[margin=1.0in]{geometry}
\usepackage{tikz-cd}

\DeclareMathOperator{\Frac}{Frac}
\DeclareMathOperator{\im}{im}
\DeclareMathOperator{\Hom}{Hom}
\DeclareMathOperator{\Ext}{Ext}
\DeclareMathOperator{\Tor}{Tor}
\DeclareMathOperator{\coker}{coker}
\DeclareMathOperator{\id}{id}
\DeclareMathOperator{\Ann}{Ann}

\newcommand{\N}{\mathbb{N}}
\newcommand{\Z}{\mathbb{Z}}
\newcommand{\R}{\mathbb{R}}
\newcommand{\C}{\mathbb{C}}

\newcommand{\tensor}{\otimes}
\newcommand{\quotient}[2]{{\raisebox{.2em}{$#1$}\left/\raisebox{-.2em}{$#2$}\right.}}
\newcommand{\isom}{\cong}

\title{Exercise sheet 9 - Bonus exercise}
\author{Matthew Dupraz}

\begin{document}
	
\maketitle

\subsection*{(1)}

Consider the following free resolution of $M$:

% https://q.uiver.app/?q=WzAsOSxbMiwwLCJSXntcXG9wbHVzMn0iXSxbMSwwLCJSIl0sWzAsMCwiMCJdLFszLDAsIk0iXSxbNCwwLCIwIl0sWzIsMSwiKGYsZykiXSxbMywxLCJ4ZiArIHlnIl0sWzEsMiwiZiJdLFsyLDIsIih5ZiwgLXhmKSJdLFsyLDFdLFsxLDBdLFswLDNdLFszLDRdLFs3LDgsIiIsMCx7InN0eWxlIjp7InRhaWwiOnsibmFtZSI6Im1hcHMgdG8ifX19XSxbNSw2LCIiLDAseyJzdHlsZSI6eyJ0YWlsIjp7Im5hbWUiOiJtYXBzIHRvIn19fV1d
\[\begin{tikzcd}[row sep=tiny]
	0 & R & R \oplus R & M & 0 \\
	&& {(f,g)} & {xf + yg} \\
	& f & {(yf, -xf)}
	\arrow[from=1-1, to=1-2]
	\arrow["\alpha", from=1-2, to=1-3]
	\arrow["\beta", from=1-3, to=1-4]
	\arrow[from=1-4, to=1-5]
	\arrow[maps to, from=3-2, to=3-3]
	\arrow[maps to, from=2-3, to=2-4]
\end{tikzcd}\]
By removing $M$ and applying the $-\tensor_R N$ functor, we get
% https://q.uiver.app/?q=WzAsNixbMiwwLCJSXntcXG9wbHVzMn1cXHRlbnNvcl9SIE4iXSxbMSwwLCJSXFx0ZW5zb3JfUiBOIl0sWzAsMCwiMCJdLFszLDAsIjAiXSxbMSwxLCJmXFx0ZW5zb3IgbiJdLFsyLDEsIih5ZiwteGYpXFx0ZW5zb3IgbiJdLFsyLDFdLFsxLDBdLFswLDNdLFs0LDUsIiIsMCx7InN0eWxlIjp7InRhaWwiOnsibmFtZSI6Im1hcHMgdG8ifX19XV0=
\[\begin{tikzcd}[row sep=tiny]
	0 & {R\tensor_R N} & {(R \oplus R)\tensor_R N} & 0 \\
	& {f\tensor n} & {(yf,-xf)\tensor n}
	\arrow[from=1-1, to=1-2]
	\arrow["{\alpha \tensor_R N}", from=1-2, to=1-3]
	\arrow[from=1-3, to=1-4]
	\arrow[maps to, from=2-2, to=2-3]
\end{tikzcd}\]
But we have that for all $f\tensor n \in R \tensor_R N$, 
\begin{align*}
	\alpha\tensor_R N(f\tensor n) &= \alpha(f)\tensor n\\
	&= (yf, -xf)\tensor n\\
	&= y(f, 0)\tensor n -x(0, f)\tensor n\\
	&= (f, 0)\tensor yn - (0, f) \tensor xn\\
	&= 0
\end{align*}
since $yn = xn = 0$ in $N = \quotient{R}{M} = \quotient{k[x, y]}{(x, y)}$.
Hence $\alpha \tensor_R N = 0$, so by calculating the homology, we get
\begin{align*}
	\Tor_0^R(M, N) &\isom (R \oplus R)\tensor_R N
	= (R\oplus R)\tensor_R \quotient{R}{M}
	\isom \quotient{R\oplus R}{M(R\oplus R)}\\
	&\isom \quotient{R}{M} \oplus \quotient{R}{M} = N\oplus N\\
	\Tor_1^R(M, N) &\isom R \tensor_R N \isom N
\end{align*}
and for all $i > 1$, we get that $\Tor_i^R(M, N) = 0$

\subsection*{(2)}

By Exercise 7, if $N$ were flat, then we would have that $\Tor_1^R(M, N) = 0$,
however that's not true by what we calculated in part (1), and so we conclude
$N$ is not flat.

\subsection*{(3)}

Consider the following free resolution of $N = \quotient{R}{M}$
\[
\begin{tikzcd}[row sep=tiny]
	0 & R & R\oplus R & R & N & 0\\
	&& (f, g) & xf + yg\\
	& f & (yf, -xf)\\
	\arrow[from=1-1, to=1-2]
	\arrow["\alpha", from=1-2, to=1-3]
	\arrow["\beta", from=1-3, to=1-4]
	\arrow["q", from=1-4, to=1-5]
	\arrow[from=1-5, to=1-6]
	\arrow[maps to, from=2-3, to=2-4]
	\arrow[maps to, from=3-2, to=3-3]
\end{tikzcd}
\]
This sequence is indeed exact, as $\im \beta = (x,y) = M = \ker q$.
Removing the $N$ and applying the $-\tensor_R N$ functor, we get
\[
\begin{tikzcd}[row sep=tiny]
	0 & R\tensor_R N & (R\oplus R)\tensor_R N & R\tensor_R N & 0\\
	&& (f, g)\tensor n & (xf + yg)\tensor n\\
	& f\tensor n & (yf, -xf)\tensor n\\
	\arrow[from=1-1, to=1-2]
	\arrow["\alpha\tensor_R N", from=1-2, to=1-3]
	\arrow["\beta\tensor_R N", from=1-3, to=1-4]
	\arrow[from=1-4, to=1-5]
	\arrow[maps to, from=2-3, to=2-4]
	\arrow[maps to, from=3-2, to=3-3]
\end{tikzcd}
\]
As before, $\alpha\tensor_R N = 0$ and also $\beta\tensor_R N = 0$, since for
any $(f, g)\tensor n \in (R\oplus R)\tensor_R N$,
\begin{align*}
	\beta\tensor_R N((f, g)\tensor n) &= \beta(f, g) \tensor n\\
	&= (xf + yg) \tensor n\\
	&= x f\tensor n + y g \tensor n\\
	&= f \tensor xn + g \tensor yn\\
	&= 0
\end{align*}
So we can calculate the homology to get
\begin{align*}
	\Tor_0^R(N, N) &\isom R \tensor_R N \isom N\\
	\Tor_1^R(N, N) &\isom (R \oplus R)\tensor_R N \isom N \oplus N\\
	\Tor_2^R(N, N) &\isom R \tensor_R N \isom N
\end{align*}
and $\Tor_i^R(N, N) = 0$ for all $i > 2$.


%\[
%\begin{tikzcd}
%	& M \tensor_R N & R \tensor_R N & N \tensor_R N & 0 \\
%	& \Tor_1^R(M, N) & \Tor_1^R(R, N) &\Tor_1^R(N, N)\\
%	\cdots & \Tor_2^R(M, N) & \Tor_2^R(R, N) &\Tor_2^R(N, N) \\
%	\arrow[from=1-2, to=1-3]
%	\arrow[from=1-3, to=1-4]
%	\arrow[from=1-4, to=1-5]
%	\arrow[from=2-2, to=2-3]
%	\arrow[from=2-3, to=2-4]
%	\arrow[from=2-4, to=1-2, out=5, in=185, looseness=1.5]
%	\arrow[from=3-2, to=3-3]
%	\arrow[from=3-3, to=3-4]
%	\arrow[from=3-4, to=2-2, out=5, in=185, looseness=1.5]
%	\arrow[from=3-1, to=3-2]
%\end{tikzcd}
%\]
\end{document}
