\documentclass{article}

\usepackage{amsmath, amssymb, amsfonts}
\usepackage[margin=1.0in]{geometry}
\usepackage{tikz-cd}

\title{Sheet 4 - Bonus exercise}
\author{Matthew Dupraz}

\newcommand{\Frac}{\mathrm{Frac}}
\newcommand{\Hom}[2]{\mathrm{Hom}_R(#1, #2)}
\newcommand{\Ext}{\mathrm{Ext}_R}
\newcommand{\im}{\mathrm{im}}
\newcommand{\Ann}{\mathrm{Ann}}

\begin{document}

\maketitle

Let $M$ be an $R$-module. We will say an element
$m \in M\setminus 0$ is \emph{divisible} if for all
$r \in R\setminus 0$,
there exists an $n \in M$, such that $m = rn$.
Hence a module is divisible iff all its elements are divisible.
	
\subsection*{(1)}

Let $M$ be a non-trivial free $R$-module, we will show that
$M$ contains no divisible elements.

Let $\{m_i\}_{i \in I}$ be a basis of $M$, let 
$r \in R\setminus 0$ which is not invertible.
Let $m \in M$.
Let $i_1, \dots, i_k \in I$ and $r_1, \dots, r_k \in R\setminus 0$
such that $m = r_1m_{i_1} + \dots + r_km_{i_k}$.
Suppose ad absurdum $m$ is divisible, then there exists
$n \in M$ such that $r_1rn = m$.
Let $j_1, \dots, j_l \in I$ and $s_1, \dots, s_l \in R\setminus 0$
such that $n = s_1m_{j_1} + \dots + s_lm_{j_l}$.
Then by the equality $r_1rn = m$ and linear independence of the
basis elements, we get that $l = k$. Furthermore, we can suppose
$i_t = j_t$ for all $t \in \{1, \dots, k\}$.
Hence we also have that
\[
	(r_1rs_1 - r_1)m_{i_1} + \dots + (r_1rs_k - r_k)m_{i_k} = 0
\]
and by linear independence of the basis elements we obtain in
particular $r_1rs_1 = r_1$ and since $M$ is an integral domain,
$rs_1 = 1$, but this is absurd, since we supposed $r$ is not
invertible. Hence $M$ contains no divisible elements, and as
a consequence is also not divisible.

\subsection*{(2)}

Suppose ad absurdum $\Frac(R)$ is projective. Then there exists
an $R$-module $M$, such that $\Frac(R)\oplus M$ is free.
We have that every element of $\Frac(R)\setminus 0$ is divisible,
since for any
$\frac{p}{q} \in \Frac(R)$, any $r \in R$, we have that
$r\frac{p}{rq}=\frac{p}{q}$.
But we have shown that a non-trivial free $R$-module contains no
divisible elements, and since
$\Frac(R) \subseteq \Frac(R)\oplus M$,
this is absurd. Hence $\Frac(R)$ is not projective.

\subsection*{(3)}

Let $M, N$ be $R$-modules. Let $P_\bullet$ be a projective
resolution of $M$ and let $\psi: N\to N$ be an $R$-module
homomorphism defined by $\psi(n) = rn$ for some fixed $r \in R$.
We have that the following diagram commutes:
\begin{equation*}
% https://tikzcd.yichuanshen.de/#N4Igdg9gJgpgziAXAbVABwnAlgFyxMJZAFgBoAGAXVJADcBDAGwFcYkRyQBfU9TXfIRRkAjNTpNW7Tjz7Y8BIgGYK4hizaIQAHW0AJCAFtgABQD65LsAByXbrxAZ5g5aTE11UrboPHzlmztZR34FIWQAJlUPSU0dfSNTMxErW3s5AUUUKPcJDXYfRPMUwPSQ5yzkEWi8r3jfJIjUoIcnTPDq3M84wr8zJtLgtrCichrugu0AYygIHAQh0JcUMa7YyZm5hfEYKABzeCJQADMAJyMkMZAcCCQmhzOLxGrr28QlYMfDJBVXpABWGL5bwJYzHZLNMpfO40G5IMi1Hqg4DgkppT7nb6IKJ-d5Auq9JIBXRobAtE6Y+Gwt6AxGTBrFKwkslQymIABs1KQAHZ8UiGf0mdpSVhySBoRyuYgABwYp683EATjlWNpcMQComIIa4IG6IebIR6s5dO1iV1kJVlylKUoXCAA
\begin{tikzcd}[row sep=large,column sep=huge]
\cdots & \Hom{P_2}{N} \arrow[d, "\Hom{P_2}{\psi}"] \arrow[l] & \Hom{P_1}{N} \arrow[d, "\Hom{P_1}{\psi}"] \arrow[l, "\Hom{f_2}{N}"] & \Hom{P_0}{N} \arrow[l, "\Hom{f_1}{N}"] \arrow[d, "\Hom{P_0}{\psi}"] & 0 \arrow[l] \arrow[d] \\
\cdots & \Hom{P_2}{N} \arrow[l]                              & \Hom{P_1}{N} \arrow[l, "\Hom{f_2}{N}"]                              & \Hom{P_0}{N} \arrow[l, "\Hom{f_1}{N}"]                              & 0 \arrow[l]          
\end{tikzcd}
\end{equation*}
Indeed, for any $i \in \mathbb{N}$, $f \in \Hom{P_i}{N}$,
\begin{equation*}
	\Hom{P_{i+1}}{\psi}\circ\Hom{f_{i+1}}{N}(f)
	= \psi \circ f \circ f_{i+1}
	= \Hom{P_{i+1}}{\psi}\circ\Hom{f_{i+1}}{N}(f)
\end{equation*}
Hence $\Hom{P_\bullet}{\psi}$ is a co-chain morphism.

Furthermore, for any $i \in \mathbb{N}$, $f \in \Hom{P_i}{N}$,
$\Hom{P_i}{\psi}(f) = \psi \circ f = rf$ and so $\Hom{P_i}{\psi}$
is also just the multiplication by $r \in R$.
By Proposition 5.4.10, $\Hom{P_\bullet}{\psi}$ induces an $R$-module
homomorphism $\Ext^i(M, \psi): \Ext^i(M, N) \to \Ext^i(M, N)$, 
$[f] \mapsto [\Hom{P_i}{\psi}(f)] = [rf] = r[f]$.
We see from the definition that $\Ext^i(M, \psi)$ is independent
of the projective resolution, and also corresponds simply to
multiplication by $r$.

\subsection*{(4)}

Fix $r \in R$, let $\phi: M \to M$ be the multiplication by $r$.
We will show that $\Ext^i(\phi, N)$ as in Definition 5.4.25
is also the multiplication by $r$ on $\Ext^i(M, N)$.
Let $P_\bullet$ be a projective resolution of $M$,
then if we let $\phi_i: P_i \to P_i, x \mapsto rx$,
we have that the following diagram commutes:
\begin{equation*}
	% https://tikzcd.yichuanshen.de/#N4Igdg9gJgpgziAXAbVABwnAlgFyxMJZARgBoAGAXVJADcBDAGwFcYkQAFAfQCYQBfUuky58hFDwrU6TVu27EBQkBmx4CRAMxSaDFm0Scu5JcLViiAFh0z97ALIAdRwGMCAcwAEACWMAKbmcAI2ZGRhgcAEpTFRF1cWRrYmk9OUMnVw8ff0DHELCI6MEzUQ0UbWTdWQMjE2LY8zLkSUrbNKNFetVShLJW1JruPi64ixRyGwH2ZxcoCBwEEcaEif7q6dc5hYFpGCh3eCJQADMAJwgAWyQJkBwIJDI2muPeGLPLh5o7pEkn9hfOsp3ldEAAOL73RDkerApAATghSAA7DDziCkYjEAA2Kp2QwvYZAtFIHG3SEAVlx7QBb2JiEpZKQ1j++OMtI+iF+30Q2hZIBedSJHK5FKpNWcaAAFlg2aiOY9uaSpoYJdKuICTnSbtyMXzVTLCZqObzuczlSB9Tt+EA
\begin{tikzcd}
\cdots \arrow[r] & P_2 \arrow[r, "f_2"] \arrow[d, "\phi_2"] & P_1 \arrow[r, "f_1"] \arrow[d, "\phi_1"] & P_0 \arrow[r, "f_0"] \arrow[d, "\phi_0"] & M\cong H_0(P_\bullet) \arrow[d, "\phi"] \\
\cdots \arrow[r] & P_2 \arrow[r, "f_2"]                     & P_1 \arrow[r, "f_1"]                     & P_0 \arrow[r, "f_0"]                     & M\cong H_0(P_\bullet)
\end{tikzcd}
\end{equation*}
Indeed, since the $f_i$ are homomorphisms, for any $x \in P_i$,
$f_i(rx) = rf_i(x)$. Hence, since such a chain is unique
up to homotopy by Proposition 5.4.20, we have that
$\Ext^i(\phi, N) := H^i(\Hom{\phi_\bullet}{N})$.
For any $i \in \mathbb{N}$, $\Hom{\phi_i}{N}$ is just the
multiplication by $r$, indeed for all $f \in \Hom{P_i}{N}$,
$x \in P_i$, we have that $\Hom{\phi_i}{N}(f)(x) = f\circ\phi_i(x)
= f(rx) = rf(x)$.
Let $[\psi] \in \Ext^i(M, N)
= \ker\Hom{f_{i+1}}{N}/\im\Hom{f_i}{N}$.
Then we have that
$H^i(\Hom{\phi_\bullet}{N})([\psi]) = [\Hom{\phi_i}{N}\psi]
= [r\psi] = r[\psi]$.]
We deduce that $\Ext^i(\phi, N) = H^i(\Hom{\phi_\bullet}{N})$
is just multiplication by $r$.

\subsection*{(5)}

Suppose $N$ is such that $\Ann(N) \neq 0$. 
We will show that $\Ext^i(\Frac(R), N) = 0$. Let $r \in \Ann(N)$,
let $\phi \in \Hom{\Frac(R)}{N}$, then for any $x \in \Frac(R)$,
we have that $\phi(x) = \phi(r\frac{x}{r})
= r\phi(\frac{x}{r}) =  0$, and hence $\Hom{\Frac(R)}{N}$ is just 
the zero module.

\end{document}
